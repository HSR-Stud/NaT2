\section{Wahrscheinlichkeitsrechnung und Zufallsvariablen \schaum{129-6}}
Dieses Kaptitel dient, ausser den Ergänzungen bei der Binominalverteilung
\verweiskurz{binominalverteilung} \normalsize, lediglich als Inhaltsverzeichnis für besseres
Zurechtfinden im Schaum.
Für detailliertere Angaben siehe WrStat-Zusammenfassung!

\subsection{Wahrscheinlichkeit \schaum{128-6.2}}
\subsubsection{Zufallsexperiment \schaum{128-6.2.A}}
\subsubsection{Zufallsraum und Ereignisse \schaum{128-6.2.B}}
\subsubsection{Ereignisalgebra \schaum{129-6.2.C}}
\subsubsection{Ereigniswahrscheinlichkeit \schaum{129-6.2.D}}
\subsubsection{Laplace Ereignis \schaum{130-6.2.E}}
\subsubsection{Bedingte Wahrscheinlichkeit \schaum{130-6.2.F}}
\subsubsection{Unabhängige Ereignisse \schaum{130-6.2.G}}
\subsubsection{Totale Wahrscheinlichkeit \schaum{131-6.2.H}}
\vspace{0.25cm}

\subsection{Zufallsvariablen \schaum{131-6.3}}
\subsubsection{Verteilungsfunktion (CDF) \schaum{132-6.3.B}}
\subsubsection{Diskrete Zufallsvariablen und diskrete Dichtefunktion (PMF, $p_X(x_i)$) \schaum{132-6.3.C}}
\subsubsection{Kontinuierliche Zufallsvariablen und kontinuerliche Dichtefunktion (PDF, $f_X(x_i)$) \schaum{132-6.3.C}}
\vspace{0.25cm}

\subsection{Zweidimensionale Zufallsvariablen \schaum{133-6.4}}

\subsubsection{Verbundsfunktionen \schaum{133,134-6.4.A,C,E}}
\vspace{-0.2cm}
\hspace*{0.2cm} Beschreiben die Abhänigkeit zwischen den beiden Zufallsvariablen
\vspace{-0.2cm}
\subsubsection{Randfunktionen \schaum{133,134-6.4.B,D,F}}
\vspace{-0.2cm}
\hspace*{0.2cm} zweidimensionale Verbundsfunktion wird durch Eliminierung einer Zufallsvariable in eine eindimensionale Funktion überführt
\vspace{-0.5cm}

\subsection{Funktionen von Zufallsvariablen \schaum{135-6.5}}
\subsubsection{Zufallsvariable - Eine Funktion von einer Zufallsvariable\schaum{135-6.5.A}}
\subsubsection{Eine Funktion von zweier Zufallsvariable \schaum{136-6.5.B}}
\subsubsection{Zwei Funktionen von zweier Zufallsvariable \schaum{136-6.5.C}}
\vspace{0.25cm}

\subsection{Statistische Mittelwerte \schaum{137-6.6}}
\subsubsection{Erwartungswert \schaum{137-6.6.A}} 
	\vspace{-0.2cm} \hspace*{0.2cm}
	\parbox{16cm}{von $Y = g(X) \Rightarrow E[Y] = E[g(X)] = \int\limits_{-\infty}^{+\infty}g(x)\cdot f_X(x) d x$ \\ 
	von $Z = g(X,Y) \Rightarrow E[Z] = E[g(X,Y)] = \int\limits_{-\infty}^{+\infty}\int\limits_{-\infty}^{+\infty} g(x,y) \cdot f_{XY}(x,y) dx \; dy$}
	\vspace{-0.2cm}
\subsubsection{Moment (n-ter Erwartungswert) \schaum{137-6.6.B}}
\subsubsection{Varianz\schaum{137-6.6.C}}
\subsubsection{Covarianz und Korrelationskoeffizent \schaum{137-6.6.D}}
\vspace{0.25cm}

\subsection{Spezielle WSK-Verteilungen \schaum{138-6.7}}
\subsubsection{Binominalverteilung\schaum{138-6.7.A}}\label{binominalverteilung}
\textbf{Approximation mit Poissionverteilung} \\
\hspace*{0.2cm} Falls $p < 0.05$ und $n > 10$, dann gilt die Approximation mit $\alpha = n \cdot p = \underbrace{\lambda}_{WrStat}$. \\
\vspace{-0.3cm}

\textbf{Approximation mit Normalverteilung} \\
\hspace*{0.2cm} Falls $n \cdot p \cdot q > 9$, dann gilt die Approximation mit $\mu = n \cdot p$ und
$\sigma^2 = n \cdot p \cdot q$.
\vspace{-0.2cm}

\subsubsection{Poissonverteilung \schaum{139-6.7.B}}
\subsubsection{Normal(Gauss-)verteilung \schaum{139-6.7.C}}
\hspace*{0.2cm} $F_X(x) = \frac{1}{2}(1+erf(\frac{x-\mu}{\sqrt{2}\sigma})) = 1-Q(\frac{x-\mu}{\sigma}) $