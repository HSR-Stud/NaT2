\section{Wahrscheinlichkeitsrechnen und Zufallsvariablen \schaum{129-6}}
Dieses Kaptitel dient, ausser den Ergänzungen bei der Binominalverteilung
\verweiskurz{binominalverteilung} \normalsize, lediglich als Inhaltsverzeichnis für besseres
Zurechtfinden im Schaum.
Für detailliertere Angaben siehe WrStat-Zusammenfassung!

\subsection{Wahrscheinlichkeit \schaum{128-6.2}}
\subsubsection{Zufallsexperiment \schaum{128-6.2.A}}
\subsubsection{Zufallsraum und Ereignisse \schaum{128-6.2.B}}
\subsubsection{Ereignisalgebra \schaum{129-6.2.C}}
\subsubsection{Ereigniswahrscheinlichkeit \schaum{129-6.2.D}}
\subsubsection{Laplace Ereignis \schaum{130-6.2.E}}
\subsubsection{Bedingte Wahrscheinlichkeit \schaum{130-6.2.F}}
\subsubsection{Unabhängige Ereignisse \schaum{130-6.2.G}}
\subsubsection{Totale Wahrscheinlichkeit \schaum{131-6.2.H}}
\vspace{0.25cm}

\subsection{Zufallsvariablen \schaum{131-6.3}}
\subsubsection{Verteilungsfunktion (CDF) \schaum{132-6.3.B}}
\subsubsection{Diskrete Zufallsvariablen und diskrete Dichtefunktion (PMF) \schaum{132-6.3.C}}
\subsubsection{Kontinuierliche Zufallsvariablen und kontinuerliche Dichtefunktion
(PDF) \schaum{132-6.3.C}}
\vspace{0.25cm}

\subsection{Zweidimensionale Zufallsvariablen \schaum{133-6.4}}
%TODO unterschied Verbundsfunktionen und Randfunktionen hervorheben
\subsubsection{Verbundsfunktionen \schaum{133,134-6.4.A,C,E}}
\subsubsection{Randfunktionen \schaum{133,134-6.4.B,D,F}}
\vspace{0.25cm}

\subsection{Funktionen von Zufallsvariablen \schaum{135-6.5}}
\subsubsection{Zufallsvariable - Eine Funktion von einer Zufallsvariable\schaum{135-6.5.A}}
\subsubsection{Eine Funktion von zweier Zufallsvariable \schaum{136-6.5.B}}
\subsubsection{Zwei Funktionen von zweier Zufallsvariable \schaum{136-6.5.C}}
\vspace{0.25cm}

\subsection{Statistische Mittelwerte \schaum{137-6.6}}
\subsubsection{Erwartungswert \schaum{137-6.6.A}}
\subsubsection{Moment (n-ter Erwartungswert) \schaum{137-6.6.B}}
\subsubsection{Varianz\schaum{137-6.6.C}}
\subsubsection{Covarianz und Korrelationskoeffizent \schaum{137-6.6.D}}
\vspace{0.25cm}

\subsection{Spezielle WSK-Verteilungen \schaum{138-6.7}}
\subsubsection{Binominalverteilung\schaum{138-6.7.A}}\label{binominalverteilung}
\textbf{Approximation mit Poissionverteilung} \\
Falls $p < 0.05$ und $n > 10$, dann gilt die Approximation mit $\alpha = n \cdot p$. \\

\textbf{Approximation mit Normalverteilung} \\
Falls $n \cdot p \cdot q > 9$, dann gilt die Approximation mit $\mu = n \cdot p$ und
$\sigma^2 = n \cdot p \cdot q$.

\subsubsection{Poissonverteilung \schaum{139-6.7.B}}
\subsubsection{Normal(Gauss-)verteilung \schaum{139-6.7.C}}