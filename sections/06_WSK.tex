\skriptsection{Wahrscheinlichkeitsrechnen und Zufallsvariablen}{129-6}
Dieses Kaptitel dient, ausser den Ergänzungen bei der Binominalverteilung
\verweiskurz{binominalverteilung} \normalsize, lediglich als Inhaltsverzeichnis für besseres
Zurechtfinden im Schaum.
Für detailliertere Angaben siehe WrStat-Zusammenfassung!

\skriptsubsection{Wahrscheinlichkeit}{128-6.2}
\skriptsubsubsection{Zufallsexperiment}{128-6.2.A}
\skriptsubsubsection{Zufallsraum und Ereignisse}{128-6.2.B}
\skriptsubsubsection{Ereignisalgebra}{129-6.2.C}
\skriptsubsubsection{Ereigniswahrscheinlichkeit}{129-6.2.D}
\skriptsubsubsection{Laplace Ereignis}{130-6.2.E}
\skriptsubsubsection{Bedingte Wahrscheinlichkeit}{130-6.2.F}
\skriptsubsubsection{Unabhängige Ereignisse}{130-6.2.G}
\skriptsubsubsection{Totale Wahrscheinlichkeit}{131-6.2.H}
\vspace{0.25cm}

\skriptsubsection{Zufallsvariablen}{131-6.3}
\skriptsubsubsection{Verteilungsfunktion (CDF)}{132-6.3.B}
\skriptsubsubsection{Diskrete Zufallsvariablen und diskrete Dichtefunktion (PMF)}{132-6.3.C}
\skriptsubsubsection{Kontinuierliche Zufallsvariablen und kontinuerliche Dichtefunktion
(PDF)}{132-6.3.C}
\vspace{0.25cm}

\skriptsubsection{Zweidimensionale Zufallsvariablen}{133-6.4}
%TODO unterschied Verbundsfunktionen und Randfunktionen hervorheben
\skriptsubsubsection{Verbundsfunktionen}{133,134-6.4.A,C,E}
\skriptsubsubsection{Randfunktionen}{133,134-6.4.B,D,F}
\vspace{0.25cm}

\skriptsubsection{Funktionen von Zufallsvariablen}{135-6.5}
\skriptsubsubsection{Zufallsvariable - Eine Funktion von einer Zufallsvariable}{135-6.5.A}
\skriptsubsubsection{Eine Funktion von zweier Zufallsvariable}{136-6.5.B}
\skriptsubsubsection{Zwei Funktionen von zweier Zufallsvariable}{136-6.5.C}
\vspace{0.25cm}

\skriptsubsection{Statistische Mittelwerte}{137-6.6}
\skriptsubsubsection{Erwartungswert}{137-6.6.A}
\skriptsubsubsection{Moment (n-ter Erwartungswert)}{137-6.6.B}
\skriptsubsubsection{Varianz}{137-6.6.C}
\skriptsubsubsection{Covarianz und Korrelationskoeffizent}{137-6.6.D}
\vspace{0.25cm}

\skriptsubsection{Spezielle WSK-Verteilungen}{138-6.7}
\skriptsubsubsection{Binominalverteilung}{138-6.7.A}\label{binominalverteilung}
\textbf{Approximation mit Poissionverteilung} \\
Falls $p < 0.05$ und $n > 10$, dann gilt die Approximation mit $\alpha = n \cdot p$. \\

\textbf{Approximation mit Normalverteilung} \\
Falls $n \cdot p \cdot q > 9$, dann gilt die Approximation mit $\mu = n \cdot p$ und
$\sigma^2 = n \cdot p \cdot q$.

\skriptsubsubsection{Poissonverteilung}{139-6.7.B}
\skriptsubsubsection{Normal(Gauss-)verteilung}{139-6.7.C}