\section{Wahrscheinlichkeitsrechnung und Zufallsvariablen \skript{117-144}}
Dieses Kaptitel dient, ausser den Ergänzungen bei der Binominalverteilung
\verweiskurz{binominalverteilung} \normalsize, lediglich als Inhaltsverzeichnis für besseres
Zurechtfinden im Skript.
Für detailliertere Angaben siehe WrStat-Zusammenfassung!
\vspace{-0.3cm}
\subsection{Wahrscheinlichkeit \skript{117-125}}
\subsubsection{Zufallsexperiment \skript{118-120}}
\subsubsection{Zufallsraum und Ereignisse \skript{119-120}}
\subsubsection{Ereigniswahrscheinlichkeit \skript{120-122}}
\subsubsection{Laplace-Ereignis \skript{122}}
\subsubsection{Bedingte Wahrscheinlichkeit \skript{122-123}}
\subsubsection{Unabhängige Ereignisse \skript{123-124}}
\subsubsection{Totale Wahrscheinlichkeit \skript{124-125}}
\vspace{0.25cm}

\subsection{Zufallsvariablen \skript{125-129}}
\subsubsection{Verteilungsfunktion (CDF) \skript{127}}
\subsubsection{Diskrete Zufallsvariable und Wahrscheinlichkeitsfunktion (PMF, $p_X(x_i)$) \skript{127-128}}
\subsubsection{Stetige Zufallsvariable und Wahrscheinlichkeitsdichtefunktion (PDF, $f_X(x_i)$) \skript{128-129}}
\vspace{0.25cm}

\subsection{Zweidimensionale Zufallsvariablen \skript{129-131}}

\subsubsection{Verbundsfunktionen \skript{129-131}}
\vspace{-0.2cm}
\hspace*{0.2cm} Beschreiben die Abhängigkeiten zwischen den beiden Zufallsvariablen.
\vspace{-\baselineskip}
\subsubsection{Randfunktionen \skript{130}}
\vspace{-0.2cm}
\hspace*{0.2cm} Zweidimensionale Verbundsfunktion wird durch Eliminierung einer Zufallsvariable in eine eindimensionale Funktion überführt.
\vspace{-2\baselineskip}

\subsection{Funktionen von Zufallsvariablen \skript{131-133}}
\subsubsection{Funktionen ein-dimensionalen Zufallsvariablen \skript{131-132}}
\subsubsection{Funktionen n-dimensionaler Zufallsvariablen \skript{132-133}}
\vspace{0.25cm}

\subsection{Statistische Kennwerte \skript{134-137}}
\subsubsection{Erwartungswert \skript{134-135}} 
	\vspace{-0.2cm} \hspace*{0.2cm}
	\parbox{16cm}{von $Y = g(X) \Rightarrow E[Y] = \int\limits_{-\infty}^{+\infty}g(x)\cdot f_X(x) d x$ \quad 
	von $Z = g(X,Y) \Rightarrow E[Z] = \int\limits_{-\infty}^{+\infty}\int\limits_{-\infty}^{+\infty} g(x,y) \cdot f_{XY}(x,y) dx \; dy$}
	\vspace{-0.2cm}
\subsubsection{Moment (n-ter Erwartungswert) \skript{135}}
\subsubsection{Varianz \skript{136}}
\subsubsection{Korrelation und Kovarianz \skript{136-137}}
\vspace{0.25cm}

\subsection{Spezielle WSK-Verteilungen \skript{138-144}}
\subsubsection{Binomialverteilung \skript{138}}\label{binominalverteilung}
\textbf{Approximation mit Poissonverteilung} \\
\hspace*{0.2cm} Falls $p < 0.05$ und $n > 10$, dann gilt die Approximation mit $\alpha = n \cdot p = \underbrace{\lambda}_{WrStat}$. \\
\vspace{-0.3cm}

\textbf{Approximation mit Normalverteilung} \\
\hspace*{0.2cm} Falls $n \cdot p \cdot q > 9$, dann gilt die Approximation mit $\mu = n \cdot p$ und
$\sigma^2 = n \cdot p \cdot q$.
\vspace{-0.2cm}

\subsubsection{Poissonverteilung \skript{138-139}}
\subsubsection{Normal(Gauss-)verteilung \skript{139}}
\hspace*{0.2cm} $F_X(x) = \frac{1}{2}(1+erf(\frac{x-\mu}{\sqrt{2}\sigma})) = 1-Q(\frac{x-\mu}{\sigma}) $