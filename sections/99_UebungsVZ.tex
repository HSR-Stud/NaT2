\section{Übungsverzeichnis}

\subsection{Wahrscheinlichkeitsrechnen und Zufallsvariablen}
	\begin{tabular}{|p{9cm}|p{2.5cm}|p{3.5cm}|p{2cm}|}
	\hline
	\textbf{Thema} & \textbf{Hausübung} & \textbf{Schaum} & \textbf{Praktikum} \\ 
	\hline
	\hline
	Allgemein & 1.1, 1.2 & 6.5, 6.7& \\
	\hline
	Satz von Bayes & 1.3 & & \\
	\hline
	Binäres Kommunikationssystem & 1.4 & 6.14 & \\
%	\hline
%	MAP (maximale a posteriori) Entscheidungstheorie & 1.5 & 6.15 & \\
	\hline
	Zufallsvariable	& 2.1 & 6.17 &  \\
	\hline
	Bitfehlerwahrscheinlichkeit binäre Datenübertragung & 2.2 & 6.20, 6.41 & \\
	\hline
	Gleichverteilung & 2.3 & 6.23 & \\
	\hline
	Exponentielle Verteilung & 2.4 & 6.25 & \\
%	\hline
%	Schütze - 2 dimensionale WSK-Verteilung & 2.5 & & \\
	\hline
	\end{tabular}
\subsection{Zufallsprozesse}
	\begin{tabular}{|p{9cm}|p{2.5cm}|p{3.5cm}|p{2cm}|}
	\hline
	\textbf{Thema} & \textbf{Hausübung} & \textbf{Schaum} & \textbf{Praktikum} \\ 
	\hline
	Autokorrelation	& & & 3.5.1 \\
	\hline
	Leistungsdichtespektrum & 4.1, 5.1 & 7.10, 7.12 & 3.7.1 \\
	\hline
	Stationarität & 3.1, 3.2, 3.3 & 7.1, 7.2, 7.4& \\
	\hline
	Ergodizität	 & 3.4 & 7.6 & \\
	\hline
	Farbige Rauschsignale & & & 3.4, 3.7.2\\
	\hline
	Zufallsprozess und LTI-System & 5.2 & 7.17 & \\
	\hline
	\end{tabular}
\subsection{Rauschen in analogen Kommunikationssystemen}
	\begin{tabular}{|p{9cm}|p{2.5cm}|p{3.5cm}|p{2cm}|}
	\hline
	\textbf{Thema} & \textbf{Hausübung} & \textbf{Schaum} & \textbf{Praktikum} \\ 
	\hline
	\hline
	Basisband & 6.1 & 8.1 & 4.3.2\\
	\hline
	Rauschleistung (Equalizer) & 6.2 & 8.2 & 3.7.3\\
	\hline
	AM (DSB-SC), Synchronisation und SNR & 6.3 & 8.3, 8.13, 8.8 & \\
	\hline
	AM (ordinary AM) & 6.4 & 8.13, 8.8 & 4.3.3\\
	\hline
	FM, SNR-Gewinn & 6.5 & 8.10, 8.13 & 4.3.4\\
	\hline
	Sendeleistung & 6.6 & & \\
	\hline
	\end{tabular}
\subsection{Optimaler Detektor - Rauschen in digitalen Kommunikationssystemen}
	\begin{tabular}{|p{9cm}|p{2.5cm}|p{3.5cm}|p{2cm}|}
	\hline
	\textbf{Thema} & \textbf{Hausübung} & \textbf{Schaum} & \textbf{Praktikum} \\ 
	\hline
	\hline
	Entscheidungsschwelle $\lambda$ & 7.1, 7.2 & 9.2, 9.3, 9.4 & \\
	\hline
	Bitfehlerwahrscheinlichkeit & 7.3 & 9.5, 9.6 & \\
	\hline
	Matched Filter & 7.4 & 9.8 &\\
	\hline
	Unmatched Filter & 7.5 & 9.9 & \\
	\hline
	\end{tabular}
\subsection{Informationstheorie und Quellencodierung}
	\begin{tabular}{|p{9cm}|p{2.5cm}|p{3.5cm}|p{2cm}|}
	\hline
	\textbf{Thema} & \textbf{Hausübung} & \textbf{Schaum} & \textbf{Praktikum} \\ 
	\hline
	\hline
	Information und Entropie & 8.1 & & \\
	\hline
	Entropie, binäre Quelle & 8.2 & & \\
	\hline
	Allgemein & & &\\
	\hline
	Binärer asymmetrischer Kanal & 9.1 & 10.7 & \\
	\hline
	Binärer symmetrischer Kanal & 9.2 & 10.16 & \\
	\hline
	AWGN-Kanal &  & 10.24 & \\
	\hline
	Shannon Fano Codierung & 9.3, 9.4, 9.5 & 10.32, 10.33, 10.34 & \\
	\hline
	Huffman Codierung & 9.4, 9.5 & 10.33, 10.34 & 6\\
	\hline
	\end{tabular}
\subsection{Kanalcodierung}
	\begin{tabular}{|p{9cm}|p{2.5cm}|p{3.5cm}|p{2cm}|}
	\hline
	\textbf{Thema} & \textbf{Hausübung} & \textbf{Schaum} & \textbf{Praktikum} \\ 
	\hline
	\hline
	Mehrfachübertragung & 10.1 & 11.3 & \\
	\hline
	Blockcodes & 9.6, 10.2, 11.1, 11.2 & 11.13 & 7.2\\
	\hline
	Zyklische Codes & 11.3, 11.4, 11.5 & & 7.3\\
	\hline
	\end{tabular}
\subsection{Sonstiges}
	\begin{tabular}{|p{9cm}|p{2.5cm}|p{3.5cm}|p{2cm}|}
	\hline  
	\textbf{Thema} & \textbf{Hausübung} & \textbf{Schaum} & \textbf{Praktikum} \\ 
	\hline
	\hline
	PLL - Phase Locked Loop	& & & 1 \\
	\hline
	8B/10B-Code & & & 2 \\
	\hline
	DTMF - Dual Tone Multiple Frequency & & & 5\\
	\hline
	\end{tabular}
