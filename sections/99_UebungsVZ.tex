\section{Übungsverzeichnis}
\vspace{-\baselineskip}
\subsection{Wahrscheinlichkeitsrechnen und Zufallsvariablen}
	\begin{tabular}{|p{8cm}|p{2.5cm}|p{1.6cm}|p{4.9cm}|}
	\hline
	\textbf{Thema} & \textbf{Hausübung} & \textbf{Praktikum} & \textbf{Prüfung} \\ 
	\hline
	\hline
	Allgemein & 1.1-1.4 \newline 2.2-2.4 & & \\
%	\hline
%	MAP (maximale a posteriori) Entscheidungstheorie & 1.5 & 6.15 & \\
	\hline
	Zufallsvariable	& 2.1 & & 02.18: 5.1 \newline 08.16: 5.1 \\
%	\hline
%	Schütze - 2 dimensionale WSK-Verteilung & 2.5 & & \\
	\hline
	\end{tabular}
\subsection{Zufallsprozesse}
	\begin{tabular}{|p{8cm}|p{2.5cm}|p{1.6cm}|p{4.9cm}|}
	\hline
	\textbf{Thema} & \textbf{Hausübung} & \textbf{Praktikum} & \textbf{Prüfung} \\ 
	\hline
	\hline
	Wahrscheinlichkeitsfunktion & & & 08.18: 1.1 \newline 02.18: 1.1, 1.2 \newline 08.17: 1.1, 1.2 \newline 02.17: 1.1, 1.2 \newline 08.16: 1.1, 1.2 \newline 08.15: 2.1 \newline 08.13: 1.1, 1.2 \\
	\hline
	Leistung & & & 08.15: 2.2, 2.3\\
	\hline
	Autokorrelation	& & 3.5.1 \newline 5 (DTMF) & 08.18: 1.2 \newline 02.18: 1.3 \newline 08.17: 1.3 \newline 02.17: 1.3 \newline 08.16: 1.3 \newline 08.15: 2.4 \newline 08.13: 1.3 \\
	\hline
	Leistungsdichtespektrum & 4.1 \newline 5.1, 5.3 & 3.7.1 & 08.18: 1.3 \newline 02.18: 1.4 \newline 08.17: 1.4, 1.5 \newline 02.17: 1.4 \newline 08.16: 1.3, 5.3 \newline 08.15: 2.5 \newline 08.13: 1.4 \\
	\hline
	Stationarität & 3.1-3.3 & & 08.15: 5.2 \newline 08.13: 5.1\\
	\hline
	Ergodizität	 & 3.4 & & \\
	\hline
	Farbige Rauschsignale & & 3.4, 3.7.2 & 02.18: 5.2, 5.3 \newline 08.17: 5.4 \newline 02.17: 5.3 \newline 08.13: 5.2\\
	\hline
	\end{tabular}
\subsection{Rauschen in analogen Kommunikationssystemen}
	\begin{tabular}{|p{8cm}|p{2.5cm}|p{1.6cm}|p{4.9cm}|}
	\hline
	\textbf{Thema} & \textbf{Hausübung} & \textbf{Praktikum} & \textbf{Prüfung} \\ 
	\hline
	\hline
	Basisband & 6.1 & 4.3.2 & 08.18: 2.1, 2.2 \newline 08.17: 5.2, 5.3 \\
	\hline
	Rauschleistung (Equalizer) & 6.2 & 3.7.3 & 08.18: 5.1 \\
	\hline
	AM (DSB-SC), Synchronisation und SNR & 6.3, 6.6 & & 02.18: 2.1, 2.3, 2.5, 2.8, 2.9 \newline 08.15: 1.1 \\
	\hline
	AM (ordinary AM) & 6.4, 6.6 & 4.3.3 & 08.15: 1.2, 5.3 \\
	\hline
	FM/PM & 6.5, 6.6 & 4.3.4 & 08.18: 2.3-2.7 \newline 02.18: 2.2, 2.4, 2.6, 2.7, 2.10 \newline 08.16: 5.4 \newline 08.15: 1.3 \\
	\hline
	\end{tabular}
\subsection{Optimaler Detektor - Rauschen in digitalen Kommunikationssystemen}
	\begin{tabular}{|p{8cm}|p{2.5cm}|p{1.6cm}|p{4.9cm}|}
	\hline
	\textbf{Thema} & \textbf{Hausübung} & \textbf{Praktikum} & \textbf{Prüfung} \\ 
	\hline
	\hline
	Entscheidungsschwelle $\lambda$ & 7.1, 7.2 & & \\
	\hline
	Bitfehlerwahrscheinlichkeit & 7.3 & & 08.17: 2.2, 2.3 \newline 08.13: 2.2-2.5 \\
	\hline
	Matched Filter & 7.4 & & 08.17: 2.1 \newline 08.13: 2.1 \\
	\hline
	Unmatched Filter & 7.5 & & \\
	\hline
	\end{tabular}
\subsection{Informationstheorie und Quellencodierung}
	\begin{tabular}{|p{8cm}|p{2.5cm}|p{1.6cm}|p{4.9cm}|}
	\hline
	\textbf{Thema} & \textbf{Hausübung} & \textbf{Praktikum} & \textbf{Prüfung} \\ 
	\hline
	\hline
	Allgemein & & & 08.18: 3.7, 3.8, 5.4 \newline 02.18: 3.1-3.4, 5.4 \newline 08.17: 2.4-2.8, 3.3, 3.4, 3.6-3.9, 5.5 \newline 08.16: 2.1-2.5, 3.1, 3.2, 5.2, 5.6 \newline 08.15: 3.8-3.10, 5.4 \newline 08.13: 3.1, 3.2, 3.4-3.6, 5.6 \\
	\hline
	Information und Entropie & 8.1, 8.2 \newline 9.1, 9.2 & & 08.18: 3.1, 3.2, 3.5 \newline 02.18: 3.5, 3.6 \newline 08.17: 3.1, 3.2 \newline 02.17: 5.5 \newline 08.16: 3.5, 3.6 \newline 08.15: 3.5-3.7 \newline 08.13: 5.5 \\
	\hline
	Kanalmatrix/-diagramm & & & 08.18: 3.6 \newline 08.15: 3.3, 3.4\\
	\hline
	AWGN-Kanal & & & 08.18: 5.2, 5.3 \newline 08.17: 5.1 \newline 02.17: 5.4 \newline 08.15: 3.1, 3.2 \newline 08.13: 5.3 \\
	\hline
	Shannon Fano Codierung & 9.3-9.5 & & 08.13: 3.3 \\
	\hline
	Huffman Codierung & 9.4, 9.5 & 6 & 08.18: 3.3, 3.4 \newline 02.18: 3.7, 3.8 \newline 08.17: 3.5 \newline 08.16: 3.3, 3.4 \newline 08.15: 5.5 \newline 08.13: 3.7 \\
	\hline
	\end{tabular}
\subsection{Kanalcodierung}
	\begin{tabular}{|p{8cm}|p{2.5cm}|p{1.6cm}|p{4.9cm}|}
	\hline
	\textbf{Thema} & \textbf{Hausübung} & \textbf{Praktikum} & \textbf{Prüfung} \\ 
	\hline
	\hline
	Allgemein & & & 08.18: 5.5 \newline 08.13: 4.1, 4.2 \\
	\hline
	Blockcodes & 9.6 \newline 10.1, 10.2 \newline 11.1, 11.2 & 7.2 & 08.18: 4.1-4.5 \newline 02.18: 4.1-4.3, 4.5-4.8, 5.5 \newline 08.17: 4.1-4.3, 5.6 \newline 02.17: 4.4-4.8, 5.6 \newline 08.16: 5.5 \newline 08.15: 4.1-4.6, 5.6 \newline 08.13: 4.1, 4.2, 4.4-4.6, 4.8-4.11\\
	\hline
	Zyklische Codes & 11.3-11.5 & 7.3 & 08.18: 4.6-4.9, 5.6 \newline 02.18: 4.4, 4.9 \newline 08.17: 4.4-4.7 \newline 02.17: 4.1-4.3 \newline 08.16: 4.1-4.7 \newline 08.15: 4.2, 4.7 \newline 08.13: 4.3, 4.7\\
	\hline
	\end{tabular}
\subsection{Sonstiges}
	\begin{tabular}{|p{8cm}|p{2.5cm}|p{1.6cm}|p{4.9cm}|}
	\hline  
	\textbf{Thema} & \textbf{Hausübung} & \textbf{Praktikum} & \textbf{Prüfung} \\ 
	\hline
	\hline
	PLL - Phase Locked Loop	& & 1 & 02.17: 5.1 \\
	\hline
	8B/10B-Code & & 2 & 02.17: 5.2 \newline 08.15: 5.1 \newline 08.13: 5.4\\
	\hline
	\end{tabular}

