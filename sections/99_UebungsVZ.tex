\section{Übungsverzeichnis}

\subsection{Wahrscheinlichkeitsrechnen und Zufallsvariablen}
	\begin{tabular}{|p{9cm}|p{2.5cm}|p{3.5cm}|p{2cm}|}
	\hline
	\textbf{Thema} & \textbf{Hausübung} & \textbf{Schaum} & \textbf{Praktikum} \\ 
	\hline
	\hline
	Allgemein & 1.1, 1.2 & 6.5, 6.7& \\
	\hline
	Satz von Bayes & 1.3 & & \\
	\hline
	Binäres Kommunikationssystem & 1.4 & 6.14 & \\
	\hline
	MAP (maximale a posteriori) Entscheidungstheorie & 1.5 & 6.15 & \\
	\hline
	Zufallsvariable	& 2.1 & 6.17 &  \\
	\hline
	Bitfehlerwahrschienlichkeit binäre Datenübertragung & 2.2 & 6.20, 6.41 & \\
	\hline
	Gleichverteilung & 2.3 & 6.23 & \\
	\hline
	Exponentielle Verteilung & 2.4 & 6.25 & \\
	\hline
	Schütze - 2 dimensionale WSK-Verteilung & 2.5 & & \\
	\hline
	\end{tabular}
\subsection{Zufallsprozesse}
	\begin{tabular}{|p{9cm}|p{2.5cm}|p{3.5cm}|p{2cm}|}
	\hline
	\textbf{Thema} & \textbf{Hausübung} & \textbf{Schaum} & \textbf{Praktikum} \\ 
	\hline
	Autokorrelation	& 4.1, 4.5 & & 2 \\
	\hline
	Leistungsdichtespektrum & 4.1, 4.2, 4.3, 4.5 & 7.10, 7.12 & 3 \\
	\hline
	Stationarität & 3.1, 3.2, 3.3 & 7.1, 7.2, 7.4& \\
	\hline
	Erdodizität	 & 3.4 & 7.6 & \\
	\hline
	Farbige Rauschsignale & & & 3-2.4\\
	\hline
	Zufallsprozess und LTI-System & 4.4 & 7.17 & \\
	\hline
	\end{tabular}
\subsection{Rauschen in analogen Kommunikationssystemen}
	\begin{tabular}{|p{9cm}|p{2.5cm}|p{3.5cm}|p{2cm}|}
	\hline
	\textbf{Thema} & \textbf{Hausübung} & \textbf{Schaum} & \textbf{Praktikum} \\ 
	\hline
	\hline
	Basisband & 5.1 & 8.1 & 4-3.1\\
	\hline
	Equalizer & 5.2 & 8.2 & \\
	\hline
	AM (DSB-SC, DSB-SSB) & 5.3, 5.6, 5.4 & 8.3, 8.13, 8.8 & \\
	\hline
	AM (ordinary AM) & 5.6, 5.4 & 8.13, 8.8 & 4-3.2\\
	\hline
	FM, PM & 5.5, 5.6 & 8.10, 8.13 & 4-3.3\\
	\hline
	\end{tabular}
\subsection{Optimaler Detektor - Rauschen in digitalen Kommunikationssystemen}
	\begin{tabular}{|p{9cm}|p{2.5cm}|p{3.5cm}|p{2cm}|}
	\hline
	\textbf{Thema} & \textbf{Hausübung} & \textbf{Schaum} & \textbf{Praktikum} \\ 
	\hline
	\hline
	Entscheidungsschwelle $\lambda$ & 6.1, 6.2 & 9.2, 9.3, 9.4 & \\
	\hline
	Bitfehlerwahrscheinlichkeit & 6.3 & 9.5, 9.6 & \\
	\hline
	Matched Filter, Korrelator & 6.4 & 9.8 & 5\\
	\hline
	Unmatched Filter & 6.5 & 9.9 & \\
	\hline
	\end{tabular}
\subsection{Informationstheorie und Quellencodierung}
	\begin{tabular}{|p{9cm}|p{2.5cm}|p{3.5cm}|p{2cm}|}
	\hline
	\textbf{Thema} & \textbf{Hausübung} & \textbf{Schaum} & \textbf{Praktikum} \\ 
	\hline
	\hline
	Allgemein & & & 6\\
	\hline
	Binärer asymmetrischer Kanal & 7.1 & 10.7 & \\
	\hline
	Binärer symmetrischer Kanal & 7.2 & 10.16 & \\
	\hline
	AWGN-Kanal & 7.3 & 10.24 & \\
	\hline
	Shannon Fano Codierung & 7.4, 7.5, 7.6 & 10.32, 10.33, 10.34 & \\
	\hline
	Huffman Codierung & 7.5, 7.6 & 10.33, 10.34 & \\
	\hline
	\end{tabular}
\subsection{Kanalcodierung}
	\begin{tabular}{|p{9cm}|p{2.5cm}|p{3.5cm}|p{2cm}|}
	\hline
	\textbf{Thema} & \textbf{Hausübung} & \textbf{Schaum} & \textbf{Praktikum} \\ 
	\hline
	\hline
	Mehrfachübertragung & 8.1 & 11.3 & \\
	\hline
	Blockcodes & 8.2 & 11.13 & 7-2\\
	\hline
	Zyklische Codes & & & 7-3\\
	\hline
	\end{tabular}
\subsection{Sonstiges}
	\begin{tabular}{|p{9cm}|p{2.5cm}|p{3.5cm}|p{2cm}|}
	\hline  
	\textbf{Thema} & \textbf{Hausübung} & \textbf{Schaum} & \textbf{Praktikum} \\ 
	\hline
	\hline
	PLL - Phase Locked Loop	& & & 1 \\
	\hline
	DTMF - Dual Tone Multiple Frequency & & & 5\\
	\hline
	\end{tabular}
