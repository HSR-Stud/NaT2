\documentclass[10pt, a4paper]{article}
\usepackage[germanb]{babel}
\usepackage{a4wide,amsmath,amssymb}
\usepackage{epsfig,array,rotating,mymacros}
\usepackage{graphicx}
\usepackage{times}
\usepackage[T1]{fontenc}
\usepackage[dvips,
            pdfauthor={Thomas Kneubuehler}
            pagebackref=true,
            colorlinks=true,
            linkcolor=blue
           ]{hyperref}
\usepackage{pdflscape} % landscape
\title{\vspace*{-4cm}\psfig{figure=elektrotechnik_sw.eps,width=6cm}
       \hfill \textit{ }\psfig{figure=ICOM_Logo_gray.eps,width=3.5cm}
                     \\[1cm]
              Nachrichtentechnik 2: Rauschen in analogen \"Ubertragungssystemen}
\author{T. Kneub\"uhler}
\date{21. April 2008}

\begin{document}
\maketitle
\noindent
%\vspace*{-15mm}


\section{Zusammenfassende Tabelle}
Um die Berechnung des Signal-Rausch Abstands in analogen \"Ubertragungssystemen zu vereinfachen, wurden einige Annahmen getroffen, welche auch in nachfolgender Tabelle Voraussetzung f\"ur die einfachen Formeln sind:
\begin{itemize}
  \item Das Nachrichtensignal (d.h. der Zufallsprozess $X(t)$) wird als 
        station\"arer Prozess (WSS) angenommen, welcher zudem noch mittelwertfrei ist.
  \item Bei den modulierten Signalen $X_{c}(t)$ soll es sich um station\"are Prozesse handeln.
        Diese Annahme ist f\"ur PM und FM unproblematisch (vgl. Schaum Aufgabe 7.1),
        f\"ur DSB-SC und AM hingegen nicht selbstverst�ndlich (vgl. Schaum Aufgabe 7.2).
        Erst mit der Einf\"uhrung einer zuf\"alligen Phase (vgl. Anmerkung 1, Schaum Kapitel 8.4.A.1)
        kann auch f\"ur DSB-SC und AM eine WSS-Stationarit\"at von $X_{c}(t)$ erzwungen werden.
  \item Der Kanal f\"ugt dem Nachrichtensignal lediglich weisses, gauss'sches Rauschen mit der 
        spektralen Leistungsdichte $\eta/2$ bei, d.h. das Nachrichtensignal wird weder linear noch
        nicht-linear verzerrt.
  \item Die Zeitverz\"ogerung beim Empf\"anger wird in dieser Tabelle vernachl\"assigt.
        So wird z.B. beim Basisbandsignal $X_{o}(t) = X_{i}(t) = X(t)$ angenommen,
        auch wenn korrekterweise $X(t-t_{d})$ stehen m\"usste.
  \item Die $SNR_{i}$ befindet sich f\"ur nachfolgende Tabelle jeweils \"uber
        dem kritischen Schwellwert, wo die $SNR_{o}$ beim Spitzenwertdetektor
        bei AM sowie auch bei PM und FM sehr schnell abf\"allt. 
        Dieser Wert liegt etwa bei $10$ dB, ist bei PM und FM aber auch stark von der Technologie des
        Empf\"angers abh\"angig (z.B. Locking-Verhalten eines PLL-Empf\"angers).
        Ein Schwellwertverhalten kann auch bei der koh\"arenten Demodulation von DSB-SC und AM
        auftreten, z.B. bei der R\"uckgewinnung des unmodulierten Tr\"agersignals mit einem PLL.
  \item Bei der koh\"arenten Demodulation wird ein lokaler Oszillator mit der Amplitude 2 benutzt,
        d.h. $x_{lo} = 2 \cos(\omega_{c}t)$. Dieser Faktor fliesst in die Berechnung der
        Ausgangsleistung $S_{o}$ bei AM und DSB-SC ein.
  
\end{itemize}

\begin{landscape}
\newpage
\begin{tabular}{|c|c|c|c|c|c|c|}
  \hline
    & Baseband 
    & DSB-SC    
    & AM Coherent 
    & AM Envelope  
    & PM  
    & FM          \\
  \hline
  Nachrichtensignal  
    & \multicolumn{6}{c|}
    {Zufallsprozess $X(t)$ mit $\left| X(t) \right| \leq 1$
     bzw. $\left| x_{\lambda}(t) \right| \leq 1$ f\"ur alle $\lambda$ des Ergebnisraums $S$} \\
  \hline
  Leistung $S_{X}$ von $X(t)$
    & \multicolumn{6}{c|}
      {$S_{X} = S_{X}(t) = E\left[ X^{2}(t)\right] \leq 1$,
      (weil $\left| X(t) \right| \leq 1$)}\\
  \hline
  Bandbreite von $X(t)$ 
    & \multicolumn{6}{c|}{B} \\
  \hline
  Eingangsnutzsignal $X_{i}(t)$
    & $X(t)$ 
    & $X(t) A_{c}\cos(\omega_{c}t)$
    & \multicolumn{2}{c|}{$A_{c}(1+\mu X(t))\cos(\omega_{c}t)$} 
    & \multicolumn{1}{c|} {$A_{c}\cos(\omega_{c}t + k_{p}X(t))$} 
    & {$A_{c}\cos(\omega_{c}t + k_{f}\int\limits_{-\infty}^{t} X(\tau)\;d\tau)$}  \\
  \hline
  Leistung $S_{i}$ von $X_{i}(t)$ 
    & $S_{X}$
    & $\frac{1}{2}A_{c}^{2} S_{X}$
    & \multicolumn{2}{c|}{$\frac{1}{2}A_{c}^{2} (1 + \mu^{2}S_{X}) $}
    & \multicolumn{1}{c|} {$\frac{1}{2}A_{c}^{2}$}
    & {$\frac{1}{2}A_{c}^{2}$} \\
  \hline
  Bandbreite von $X_{i}(t)$ 
    & $B$
    & $2B$
    & \multicolumn{2}{c|}{$2B$}
    & \multicolumn{1}{c|}{$2(D + 1) B$}
    & {$2(D + 1) B$} \\
  \hline
  Rauschleistung am Eingang
    & $\eta B$
    & $2\eta B$
    & \multicolumn{2}{c|}{$2\eta B$}
    & \multicolumn{1}{c|}{$2(D + 1)\eta B$}
    & {$2(D + 1)\eta B$} \\
  \hline
  SNR am Eingang $\left(\frac{S}{N}\right)_{i}$
    & $\frac{S_{i}}{\eta B}$
    & $\frac{\frac{1}{2}A_{c}^{2} S_{X}}{2\eta B}$
    & \multicolumn{2}{c|}{$\frac{\frac{1}{2}A_{c}^{2} (1 + \mu^{2}S_{X})}{2\eta B}$}
    & \multicolumn{1}{c|}{$\frac{\frac{1}{2}A_{c}^{2}}{2(D + 1)\eta B}$}
    & {$\frac{\frac{1}{2}A_{c}^{2}}{2(D + 1)\eta B}$} \\
  \hline
  Ausgangsnutzsignal $X_{o}(t)$
    & $X(t)$ 
    & $A_{c}X(t)$
    & \multicolumn{2}{c|}{$A_{c}\mu X(t)$} 
    & \multicolumn{1}{c|} {$k_{p}X(t)$} 
    & {$k_{f}X(t)$}  \\
  \hline
  Leistung $S_{o}$ von $X_{o}(t)$   
    & $S_{X}$
    & $A_{c}^{2} S_{X}$
    & \multicolumn{2}{c|}{$A_{c}^{2}\mu^{2}S_{X}$}
    & \multicolumn{1}{c|} {$k_{p}^{2}S_{X}$}
    & {$k_{f}^{2}S_{X}$} \\
  \hline
  Rauschleistung am Ausgang
    & $\eta B$
    & $2\eta B$
    & \multicolumn{2}{c|}{$2\eta B$}
    & \multicolumn{1}{c|}{$\frac{1}{A_{c}^{2}/2} \eta B$}
    & {$\frac{1}{3}\frac{(2\pi B)^{2}}{A_{c}^{2}/2} \eta B$} \\
  \hline
  SNR am Ausgang $\left(\frac{S}{N}\right)_{o}$
    & $\frac{S_{i}}{\eta B}$
    & $\frac{A_{c}^{2} S_{X}}{2\eta B}$
    & \multicolumn{2}{c|}{$\frac{A_{c}^{2}\mu^{2}S_{X}}{2\eta B}$}
    & \multicolumn{1}{c|}{$\frac{k_{p}^{2}A_{c}^{2}S_{X}}{2\eta B}$}
    & {$\frac{3 D^{2}A_{c}^{2}S_{X}}{2\eta B}$} \\
  \hline
  $\left(\frac{S}{N}\right)_{o}$ ausgedr\"uckt mit  $\gamma = \frac{S_{i}}{\eta B}$
    & $\gamma$
    & $\gamma$
    & \multicolumn{2}{c|}{$\frac{\mu^{2}S_{X}}{1 + \mu^{2}S_{X}}\gamma$}
    & \multicolumn{1}{c|}{$k_{p}^{2}S_{X}\gamma$}
    & {$3 D^{2}S_{X}\gamma$} \\
  \hline 
\end{tabular}
\\[1cm]
Wichtige Anmerkung: Die Formeln der Tabelle gelten f\"ur dimensionslose Signale. Der Zufallsprozess liegt zudem in normierter Form vor, wie aus der Tabelle hervorgeht. Soll die SNR f�r konkrete physikalisch vorliegende Signale berechnet werden, m\"ussen f\"ur die Amplituden und Leistungen am Eingang des Empf\"angers geeignete Saklierungsfaktoren verwendet werden. Handelt es sich beim Empf\"anger zudem um einen aktiven Schaltungsblock, ist das Signal (sowie der Rauschanteil) am Ausgang des Empf\"angers ebenfalls mit den Parametern des Empf\"angers zu skalieren. \\
\end{landscape}

\end{document}
 
